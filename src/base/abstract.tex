% This file contains the abstract of the thesis

Malware is currently a major security threat for computers and smartphones, with efforts being taken into improving malware detectors with behavior-based detection. In order to classify applications, malware detectors need some form of malicios behavior specification which are usually identified manually by researchers. We present a Linux implementation of the malspec-mining algorithm which automates this process. This algorithm recognizes such specifications by comparing known malicious and benign applications. The output consists of behavior patterns which are specific to the inputted malware and that do not occur in benign applications.

\textbf{Keywords:} behavior-based detection; malspec-mining algorithm; malicious behavior; kernel programming

\vspace*{2cm}

În prezent, aplicaţiile de tip malware reprezintă o ameninţare majoră pentru securitatea calculatoarelor şi a smartphone-urilor, luându-se măsuri pentru îmbunătăţirea detectoarelor de malware bazate pe comportament. Pentru a putea clasifica aplicaţii, detectoarele de malware au nevoie de specificaţii ale comportamentului maliţios, care sunt de obicei identificate manual de către cercetători. Vom prezenta o implementare pe Linux a algoritmului de malspec-mining care automatizează acest proces. Acest algoritm identifică astfel de specificaţii prin compararea de aplicaţii cunoscute ca fiind maliţioase şi aplicaţii benigne. Rezultatul constă în tipare de comportament care sunt specifice malware-ului şi care nu apar în aplicaţiile benigne.

\textbf{Cuvinte cheie:} detecţie bazată pe comportament; algoritmul de malspec-mining; comportament maliţios; programare kernel