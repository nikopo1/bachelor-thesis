\chapter{Conclusions and Future Work}
\label{chapter:sixth}

Computer security is an important field and continuous advances ensure protection from threats such as disruption of normal computer and network operation, information leaks and attackers who benefit financially from compromised hosts.

As Linux based operating systems are becoming evermore popular, malware security for these systems will continue to grow in importance. Also, use in embedded systems makes Linux an important target for malware writers.

In this thesis, we propose a solution for automating the dynamic analysis of malware samples, called \textbf{\project}, which implements the malware specification mining algorithm. It compares malicious and benign programs and identifies behavior patterns specific to malware.

Our implementation of the malspec mining algorithm successfully collected the execution traces of all the malware samples. The dependency graphs that it generated were correct and the edges connected only nodes that had def-use dependencies.

As shown by our evaluation results, the node aggregation we implemented reduces the number of nodes in large graphs by a considerable factor which resulted in a shorter execution time.

The XML configuration file we used for decoding system call argument type and direction can be easily modified to include different sets of system calls from the ones we used. Also, by omitting registers from this file a selective decoding pattern can be implemented which uses only particular system call arguments for dependency graph generation.

The \textbf{\project}\ tool represents a proof of concept for a Linux implementation of the malware specification mining algorithm. The malware specifications it identified demonstrate that this algorithm applies for Linux based operating systems and that further improvements will certainly improve the results that we obtained so far.

First, the traces are currently obtained by using the system call interceptor driver. In the future we plan to switch to using strace instead. Also, using iptables instead of the network interceptor module might be a possible improvement.

Secondly, another possible improvement includes adding other dependency edges apart from def-use, such as: comparing strings for common substrings or self-referential dependencies for when the malware sample opens its own file.

Finally, the backtracking algorithm for determining the maximal common subgraph can be further improved with other pruning methods like node ordering strategies and ancestor-leaf equivalence pruning.

\textbf{\project}\ will be released under an Open Source license to permit others to contribute to this project. Work will be continued on this project and we believe it will become a powerful and reliable tool for Linux malware analysts.