\chapter{Conclusions and Future Work}
\label{chapter:sixth-chapter}

Computer security is an important field and continuous advances ensure protection from threats such as disruption of normal computer and network operation, information leaks and attackers who benefit financially from compromised hosts.

As Linux based operating systems are becoming evermore popular, malware security for these systems will grow in importance. Also, use in embedded systems make Linux an important target for malware writers.

The \textbf{\project}\ tool represents a proof of concept for a Linux implementation of the malware specification mining algorithm and leaves room for developing a more advanced detector. There are still a lot of possible improvements that are planned for implementation.

First, the traces are currently obtained by using the system call interceptor driver. In the future we plan to switch to using strace instead. Also, using iptables instead of the network interceptor module might be a possible improvement.

Secondly, another possible improvement includes adding other dependency edges apart from def-use, such as: comparing strings for common substrings or self-referential dependencies for when the malware file opens its own file.

Finally, the backtracking algorithm for determining the maximal common subgraph can be further improved with other prunning methods like node ordering strategies and ancestor-leaf equivalence prunning.

\textbf{\project}\ will be released under an Open Source license to permit others to contibute to this project. We hope that with time, this will become a useful and reliable tool for Linux malware analysts.