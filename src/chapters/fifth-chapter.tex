\chapter{Malsharp Evaluation}
\label{chapter:fifth}

In this chapter we will present our evaluation for \project. We begin by describing the test environment we used for analyzing malware in \labelindexref{section}{fifth:test-env}. Then we describe a typical test scenario and we use a short example to illustrate how the malspec mining algorithm works on a small graph in \labelindexref{section}{fifth:test-scenario}. Finally, we present our evaluation results for a set of malware samples in \labelindexref{section}{fifth:evaluation-results}.

\section{Malware test environment}
\label{fifth:test-env}

\project\ was tested in a sandbox environment: a VMware virtual machine with a distribution of Ubuntu with the Linux 3.7.8 kernel which has been bridged to the local network for Internet access.

To ensure that all the malware samples were tested in the same environment, we used snapshots so that each file was tested on the same virtual machine state. In this way, different malware samples could not influence each other's execution.

\fig[scale=1.0]{src/img/test-env.png}{img:test-env}{Virtual Machine test configuration}

In \labelindexref{Figure}{img:test-env} the snapshot test configuration is shown. The ``Clean malware state'' is the malware test snapshot we mentioned earlier. ``Before cutting off for malware test'' is the VM\abbrev{VM}{Virtual Machine} state before changing the connection type from NAT\abbrev{NAT}{Network Address Translation} to bridge and before removing all ssh keys from the machine.

\section{Test scenario}
\label{fifth:test-scenario}

In a typical test scenario, \project\ receives an XML file which contains the system calls with parameter information that we want to monitor in our executables. We can test for malware specifications only on a subset of the total system calls a Linux operating system has.

For a better comprehension on how the malspec mining algorithm works, we will consider the following example. Let us assume that we want to search for malware specifications with \code{open}, \code{read}, \code{write} and \code{close} system calls. In this example, \code{program_test} will be the malware sample and \code{diff_test} will be the benign program.

The following table presents the system calls that each executable file makes and the relevant argument information for the dependency edges.

\begin{center}
\begin{table}[htb]
  \caption{System calls for program_test and diff_test}
  \begin{center}
  \code{
  \begin{tabular}{cl*{6}l}
     & program_test & diff_test \\
    \hline
    1 & open(...) = fd1 & open(...) = fd1  \\
    2 & close(fd1) & close(fd1)  \\
    3 & open(...) = fd2 & open(...) = fd2 \\
    4 & read(fd2, ...) & read(fd2, ...)  \\
    5 & close(fd2) & close(fd2) \\
    6 & open(...) = fd3 & open(...) = fd3\\
    7 & write(fd3, ...) & close(fd3) \\
    8 & write(fd3, ...) & open(...) = fd4 \\
    9 & write(fd3, ...) & write(fd4, ...) \\
    10 & close(fd3) & read(fd4, ...) \\
    11 & \multicolumn{1}{c}{-} & close(fd4)
  \end{tabular}
  }
  \end{center}
  \label{table:test-programs}
\end{table}
\end{center}

\fig[scale=0.65]{src/img/max-common-edge-set.pdf}{img:max-common-edge-set}{Maximal common edge set. (a) program_test; (b) diff_test.}

In \labelindexref{Figure}{img:max-common-edge-set} the generated graphs for the two programs presented in \labelindexref{Table}{table:test-programs} are shown. For each node we have written its index in the graph and its label.

Although there are three \code{write} calls in \code{program_test} that operate on the same file descriptor, these nodes are not aggregated because of the fourth restriction we added to node aggregation in \labelindexref{Section}{fourth:dep-graphs}. The three nodes cannot be aggregated because of the edges that exist between them.

The maximal common edge set that was determined using McGregor's algorithm is shown by drawing paired nodes with the same colors. The nodes that do not have a matching node in the other graph have no color. The edges connecting them are gray because they are not a part of the maximal common edge set.

After computing the maximal common edge set, its complement in the malware graph is computed. In this particular example, the complement is a single connected graph, so the minimal union step of the algorithm is not required.

Then, a breadth first search is used to find the minimal transversal of the complement. The result of these two steps of the algorithm can be seen in \labelindexref{Figure}{img:min-transversal-compl}. The malspec contains two nodes which are colored red and the black edge connecting them. Because we used a single benign program to find the malspec, this will be the final result of the algorithm.

\fig[scale=0.65]{src/img/min-transversal-compl.pdf}{img:min-transversal-compl}{Minimal transversal of complement graph. (a) program_test; (b) diff_test.}

The complete output from \project\ for \code{program_text} and \code{diff_test} can be seen in \labelindexref{Appendix}{lst:output}.

\section{Evaluation results}
\label{fifth:evaluation-results}

\project\ was tested using the Linux malware samples provided by \cite{open-malware}. All the malware samples that were tested also had additional classification information provided by antivirus software.

As pointed out by Burguers \textit{et al.} in \cite{crowdroid-malware-android}, the increasing number of smartphones on the market with the Android platform, which has a Linux kernel, make malware analysis an important issue.

Therefore, we used two sets of system calls to monitor the malicious and benign programs: a smaller set which was indicated by Burguers \textit{et al.} in their paper about behavior-based detection on Android \cite{crowdroid-malware-android} and a larger set which contained other system calls that might be used for malicious purposes.

The larger set is a superset of the other and has more system calls that use file descriptors and perform socket communication. Only a part of this XML file has been added to this paper as \labelindexref{Appendix}{lst:syscalls-xml} because the original file was two large.

The results for each set of system calls\abbrev{syscall}{system call} can be seen in the following two tables, one per set. The set of benign programs we considered for testing consisted of: \code{ls}, \code{lsmod}, \code{ping} and \code{cat}.

In \labelindexref{Table}{table:mal-analysis-android} and \labelindexref{Table}{table:mal-analysis-all}, we used the following notation:

\begin{itemize}
    \item N - number of nodes of the dependency graph
    \item N' - number of nodes left after node aggregation
    \item E - number of edges of the dependency graph
\end{itemize}

\begin{center}
\begin{table}[htb]
  \caption{Malware sample analysis for android.xml}
  \begin{center}
  \begin{tabular}{lcccccl}
    Sample & N & N' & E & Malspecs & Time & Observed behavior\\
    \hline
    Backdoor.Linux.CGI.a   & 12  & 12  & 5   & 1 & 10.208s & \code{open, read, close} \\
    Backdoor.Linux.Phobi.1 & 36  & 35  & 16  & 1 & 0.0641s & \code{open, read, close} \\
    Trojan.Linux.Rootkit.n & 12  & 11  & 5   & 1 & 0.558s  & \code{open, read, close} \\
    Virus.Linux.Osf.8759   & 692 & 442 & 285 & 0 & \tl     & \tl \\
    Virus.Linux.Radix      & 14  & 13  & 4   & 1 & 0.786s  & \code{open, read} \\
    Virus.Linux.Rike.1627  & 47  & 47  & 21  & 1 & 0.674s  & \code{open, read, close} \\
    Virus.Linux.Silvio.b   & 46  & 46  & 24  & 1 & 0.983s  & \code{open, read, close} \\
    Virus.Linux.Snoopy.c   & 128 & 93  & 39  & 1 & 1.271s  & \code{open, read, close} \\
    Virus.Linux.Svat.b     & 23  & 23  & 12  & 0 & 0.617s  & \tl \\
  \end{tabular}
  \end{center}
  \label{table:mal-analysis-android}
\end{table}
\end{center}

In \labelindexref{Table}{table:mal-analysis-android}, we show the results for 9 malware samples. Although the initial collection of malware samples had 20 samples, some samples could not be tested because the operating system on our VM did not have all the shared libraries they required in order to run. Also, other samples were actually tools for flooding a range of ip addresses in a network.

\code{Virus.Linux.Osf.8759} could not be compared to the set of benign programs because it infected the benign programs. Due to the fact that this particular sample inserted itself into the benign programs before we extracted their trace, the analysis was not relevant because the programs already contained malicious behavior. Also, it had a specific feature from the other malware samples, it accessed the system call interceptor driver.

We compared the system calls that the \code{ls} command, which is part of the benign program set, made before and after the \code{Virus.Linux.Osf.8759} sample was run. Although, in appearance, the behavior of the command was the same, in fact it made additional calls to \code{fork}, \code{socket}, \code{listen}, \code{bind} and \code{pipe}. These system calls were not a part of the initial program trace before the sample was run.

\code{Virus.Linux.Osf.8759} generated the biggest graph and we can see that the node aggregation method reduced the total number of nodes considerably, from 692 to 442. Although analysis was not possible, the node aggregation method is done separately and the results are valid.

Most of the samples generated dependency graphs with a very small number of nodes. This is a consequence of the fact that android.xml contains a very small number of system calls used for monitoring: \code{read}, \code{open}, \code{close}, \code{chmod}, \code{lchown} and \code{access}.

\begin{center}
\begin{table}[htb]
  \caption{Malware sample analysis for all_syscalls.xml}
  \begin{center}
  \begin{tabular}{lcccccl}
    Sample & N & N' & E & Malspecs & Time & Observed behavior \\
    \hline
    Backdoor.Linux.CGI.a   & 13  & 13  & 5   & 1 & 0.827s  & \code{open, read, close}  \\
    Backdoor.Linux.Phobi.1 & 36  & 35  & 16  & 1 & 0.641s  & \code{open, read, close}  \\
    Trojan.Linux.Rootkit.n & 14  & 14  & 5   & 1 & 1.044s  & \code{open, read, close}  \\
    Virus.Linux.Osf.8759   & 19  & 19  & 9   & 0 & \tl     & \tl \\
    Virus.Linux.Radix      & 25  & 22  & 6   & 1 & 0.673s  & \code{open, read, write;} \\ 
                           &     &     &     &   &         & \code{creat, write}       \\
    Virus.Linux.Svat.b     & 25  & 25  & 12  & 0 & 3.495s  & replaces \code{stdio.h}   \\
  \end{tabular}
  \end{center}
  \label{table:mal-analysis-all}
\end{table}
\end{center}

Analysis results for the larger set of system calls is shown in \labelindexref{Table}{table:mal-analysis-all}. The total analysis took longer due to the fact that the generated dependency graphs were larger, especially in the case of the benign programs.

Our implementation succeeded in obtaining malspecs for the analyzed malware samples. The most remarcable result was for the \code{Virus.Linux.Radix} sample, although it had a shorter analysis time in comparison to other samples. The malspec it returned consisted of two graphs.

The first graph contains the following nodes: \code{open}, \code{read}, \code{write}. The \code{open} node had descriptor dependencies with the other two calls. The second graph contained two nodes: a \code{creat} system call and a call to \code{write}.

\code{Virus.Linux.Radix} also presented another interesting behavior pattern, it inserted itself into the executable file for the \code{ls} command, \code{/bin/ls}, and \code{cp} command, \code{/bin/cp}. It also searched the current directory for executable files and it infected those as well.

\fig[scale=0.58]{src/img/node-aggregation-android.png}{img:node-aggregation-android}{Node aggregation for android.xml}

Node aggregation reduced the total number of nodes for the dependency graphs, especially in the case of large graphs. In \labelindexref{Figure}{img:node-aggregation-android} we can see the total number of nodes was reduced by a significant factor for \code{Virus.Linux.Osf.8759} and \code{Virus.linux.Snoopy.c}.

In \labelindexref{Figure}{img:mal-analysis-time} we present the analysis duration for the malware samples we considered for the android.xml test.

\fig[scale=0.58]{src/img/mal-analysis-time.png}{img:mal-analysis-time}{Analysis time for android.xml samples}

\labelindexref{Figure}{img:mal-memory-use} shows the memory use of \project\ for the android.xml test. Memory usage was measured by using \textbf{Valgrind} while also checking for potential memory leaks. 

For each malware sample, the blue area, drawn according to the left vertical axis, shows how much memory was used by \project\ while the red area, drawn according to the right vertical axis, represents the number of nodes in the dependency graph.

It can be observed that the memory usage is proportional to the number of nodes of each malware sample. This was the expected result considering that all the malware samples were tested against the same benign programs which had a constant number of system calls.

\fig[scale=0.47]{src/img/mal-memory-use.png}{img:mal-memory-use}{Malsharp memory use for android.xml test}